\documentclass{report}
\usepackage{subfiles}
\usepackage{algorithm}
\usepackage{algpseudocode}
\usepackage{hyperref}
\usepackage[utf8]{inputenc}
\usepackage{subfiles}
\usepackage{tikz}
\usepackage{color}
\usepackage{amsmath}
\usepackage{upgreek}
\usetikzlibrary{arrows.meta}
\usepackage[utf8]{inputenc}
\usepackage{subfiles}
\usepackage{tikz}
\usepackage{amssymb}
\usepackage[utf8]{inputenc}
\usepackage{algorithm}
\usepackage{algpseudocode}
%\usepackage[boxruled,vlined,linesnumbered]{algorithm2e}
\usepackage{hyperref}
\usepackage[utf8]{inputenc}
\usepackage{subfiles}
\usepackage{tikz}
\usepackage{color}
\usepackage{amsmath}
\usepackage{upgreek}
\usetikzlibrary{arrows.meta}
\usepackage[utf8]{inputenc}
\usepackage{subfiles}
\usepackage{tikz}
\usepackage{array}
%Color begin
%\definecolor{lineColor}{rgb}{0, 0.173, 0.95 }
\definecolor{lineColor}{rgb}{0.0235, 0.941, 0.196 } %green colored line..
\definecolor{nodeWriter}{rgb}{0, 0, 0}

\definecolor{background}{rgb}{0.84, 0.92, 0.95}
\definecolor{myCyan}{rgb}{0.11, 0.71, 0.86}
\definecolor{myPurple}{rgb}{0.64, 0.46, 0.66}
\definecolor{myLightPurple}{rgb}{0.835, 0.686, 0.894}
\definecolor{myOrange}{rgb}{0.96, 0.66, 0.59}
\definecolor{textColor}{rgb}{0.31,0.20,0.53}
%\definecolor{lineColor}{rgb}{0.68,0.45,0.45}
\definecolor{myRed}{rgb}{0.964, 0.172, 0.172}
\definecolor{circleBorder}{rgb}{0.09,0.17,0.39}
\definecolor{diameterColor}{rgb}{0.22,0.29,0.42}
\definecolor{myLightBlue}{rgb}{0.76,0.84,0.95}
\definecolor{myGreen}{rgb}{0.098, 0.709, 0.090}

\definecolor{whiteColor}{rgb}{1,1,1}


\definecolor{textColor}{rgb}{0.0235, 0, 0.85 }    %apatoto myGreen tai

\definecolor{superScriptColor}{rgb}{0.964, 0.172, 0.172}   %apatoto myRed

%Color end

%begin of document
\begin{document}
%titlepage
\begin{titlepage}

\newcommand{\HRule}{\rule{\linewidth}{0.5mm}} % Defines a new command for the horizontal lines, change thickness here

\center % Center everything on the page
 
%----------------------------------------------------------------------------------------
%	HEADING SECTIONS
%----------------------------------------------------------------------------------------

\textsc{\large Bangladesh University of Engineering and Technology}\\[1.5cm] % Name of your university/college
\textsc{\Large CSE 300}\\[0.5cm] % Major heading such as course name
\textsc{\large Technical Writing And Presentation}\\[0.5cm] % Minor heading such as course title

%----------------------------------------------------------------------------------------
%	TITLE SECTION
%----------------------------------------------------------------------------------------

\HRule \\[0.4cm]
{ \huge \bfseries A Report on Priority Queue}\\[0.4cm] % Title of your document
\HRule \\[1.5cm]
 
%----------------------------------------------------------------------------------------
%	AUTHOR SECTION
%----------------------------------------------------------------------------------------

\begin{minipage}{0.4\textwidth}
\begin{flushleft} \large
\emph{Author:}\\
Mahim Mahbub\\
Kishore Kumar Dash
\end{flushleft}
\end{minipage}
~
\begin{minipage}{0.4\textwidth}
\begin{flushright} \large
\emph{Student ID:} \\
1505022\\
1505028% Supervisor's Name
\end{flushright}
\end{minipage}\\[2cm]

% If you don't want a supervisor, uncomment the two lines below and remove the section above
%\Large \emph{Author:}\\
%John \textsc{Smith}\\[3cm] % Your name

%----------------------------------------------------------------------------------------
%	DATE SECTION
%----------------------------------------------------------------------------------------

{\large \today}\\[2cm] % Date, change the \today to a set date if you want to be precise

%----------------------------------------------------------------------------------------
%	LOGO SECTION
%----------------------------------------------------------------------------------------

    \vfill
    \includegraphics[width=4cm]{runner/BUET.png}% also works with logo.pdf
    \vfill
    \vfill
\end{titlepage}


%introduction

%runner part begins

\newpage 

\tableofcontents

\chapter{Priority Queue}
\section{Introduction}
\subfile{runner/intro.tex}

\newpage
\section{What is a Priority Queue?}
\subfile{runner/definition.tex}

\newpage
\section{Functionalities of Priority Queue}
\subfile{runner/functionalities.tex}

\newpage
\section{How to implement Priority Queue?}
\subfile{runner/howToImpelement.tex}

\newpage
\section{Binary Heap}
\subfile{runner/binaryHeap.tex}

\newpage
\section{Implementation using Max Heap}
\subfile{runner/maxHeapmaxPriorityQueue.tex}

\newpage
\section{Sample Simulations}
Given the algorithms above, let us take a look at a few simulations of those functions.
\subsection{Simulation for Max\textunderscore Heapify}

\subfile{runner/simulations/heapify.tex}

\newpage

\subsection{Simulation for IncreaseKey(A, x, newKey)}
\subfile{runner/simulations2/increaseKey.tex}

\newpage
\section{Pros and Cons of using Binary Heap to implement Priority Queue}
\subfile{runner/prosAndCons.tex}

\newpage
\section{Conclusion}
\subfile{runner/conclusion.tex}

%runner part ends

\end{document}







