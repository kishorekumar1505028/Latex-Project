Just like any other interface, priority queue has a few \textbf{signature} functions.
For a Maximum Priority Queue, the signature functions are as follows.
\subsection{Functions Of Maximum Priority Queue}
\begin{enumerate}
    \item \textbf{Insert(S, x)} : Inserts a new element with priority x into the set of elements S, which was already a priority queue.
    \item \textbf{Maximum(S)} : Takes a peek and returns (but does not remove) at the element with the maximum priority in the priority queue.
    \item \textbf{ExtractMax(S)}: Returns and removes the element with the maximum 
    priority in the priority queue.
    \item \textbf{IncreaseKey(S, x, newKey)}: Increases the key of this current element and moves its place accordingly in the priority queue.
\end{enumerate}

\subsection{Functions Of Minimum Priority Queue}
For a Minimum Priority Queue, the signature functions are as follows.
\begin{enumerate}
    \item \textbf{Insert(S, x)}: Inserts a new element with priority x into the set of elements S, which was already a priority queue.
    \item \textbf{Minimum(S)} : Takes a peek and returns (but does not remove) at the element with the minimum priority in the priority queue.
    \item \textbf{ExtractMin(S)}: Returns and removes the element with the minimum 
    priority in the priority queue.
    \item \textbf{DecreaseKey(S, x, newKey)}: Decreases the key of this current element and moves its place accordingly in the priority queue.
\end{enumerate}
