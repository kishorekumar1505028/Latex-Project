So far we have learned what a Priority Queue is and how to efficiently implement it.
We have also come across various real life examples of where a priority queue interfaces are used.
We have learned the signature functions of the priority queue and also how to implement those functions if a binary heap were used to implement the priority queue interface itself.


Priority Queues are used in many algorithms too for example a minimum priority queue is used in Dijkstra's Algorithm, Huffman's Compression Algorithm, etc.

These are also used in processor scheduling of tasks. In airport scheduling of air-crafts and the list goes on.

We can now conclude on the remark that priority queues are very useful in Computer Science and the interface is relatively very easy to understand and simple to
implement.