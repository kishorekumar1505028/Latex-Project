In computer science, a priority queue is an abstract data type which is like a regular queue or stack data structure, but where additionally each element has a \textbf{priority} associated with it. 
\newline
There are two types of priority queues: Maximum Priority Queue and Minimum Priority Queue.
\subsection{Different Types Of Priority Queues}
There are two different types of priority queues with respect to which priority we
want to concern ourselves with.
\begin{enumerate}
    \item \textbf{Maximum Priority Queue}: An element with higher priority is served before an element with lower priority.
    \item \textbf{Minimum Priority Queue}: An element with lower priority is served before an element with higher priority.
\end{enumerate}

\subsection{Applications}
Priority queue is used extensively in Computer Science and here are a few of its 
applications.
\begin{itemize}
    \item Operating System Design resource allocation.
    \item Minimum Priority Queue is used in Data Compression - Huffman algorithm.
    \item Discrete Event simulation
    \begin{enumerate}
        \item Insertion of time-tagged events (time represents a priority of an event low time means high priority).
        \item Removal of the event with the smallest time tag.
    \end{enumerate}
\end{itemize}

