To introduce a priority queue, let us take a quick recap of the basic data structures queue and stack.
\subsection{Queue}
A queue is an interface which stores elements that are accessed in first-in-first-out way. The best example is in a shopping mall where if you go first to the cashier, you are served first.
\subsection{Stack}
A stack is an interface which stores elements but unlike a queue, those are accessed in first-in-last-out way. The best example is an elevator where if you are the last 
person to enter, you have the best chance to get out the earliest.
\subsection{Priority Queue}
But for priority queue, each elements has an additional property called \textbf{priority} and this priority is used when we want to access an element first in
the collection of elements.

\subfile{runner/example.tex}


